\documentclass[10pt, letterpaper]{article}

% Packages:
\usepackage[
    ignoreheadfoot, % set margins without considering header and footer
    top=2 cm, % seperation between body and page edge from the top
    bottom=2 cm, % seperation between body and page edge from the bottom
    left=2 cm, % seperation between body and page edge from the left
    right=2 cm, % seperation between body and page edge from the right
    footskip=1.0 cm, % seperation between body and footer
    % showframe % for debugging 
]{geometry} % for adjusting page geometry
\usepackage{titlesec} % for customizing section titles
\usepackage{tabularx} % for making tables with fixed width columns
\usepackage{array} % tabularx requires this
\usepackage[dvipsnames]{xcolor} % for coloring text
\definecolor{primaryColor}{RGB}{0, 0, 0} % define primary color
\usepackage{enumitem} % for customizing lists
\usepackage{fontawesome5} % for using icons
\usepackage{amsmath} % for math
\usepackage[
    pdftitle={John Doe's CV},
    pdfauthor={John Doe},
    pdfcreator={LaTeX with RenderCV},
    colorlinks=true,
    urlcolor=primaryColor
]{hyperref} % for links, metadata and bookmarks
\usepackage[pscoord]{eso-pic} % for floating text on the page
\usepackage{calc} % for calculating lengths
\usepackage{bookmark} % for bookmarks
\usepackage{lastpage} % for getting the total number of pages
\usepackage{changepage} % for one column entries (adjustwidth environment)
\usepackage{paracol} % for two and three column entries
\usepackage{ifthen} % for conditional statements
\usepackage{needspace} % for avoiding page brake right after the section title
\usepackage{iftex} % check if engine is pdflatex, xetex or luatex

% Ensure that generate pdf is machine readable/ATS parsable:
\ifPDFTeX
    \input{glyphtounicode}
    \pdfgentounicode=1
    \usepackage[T1]{fontenc}
    \usepackage[utf8]{inputenc}
    \usepackage{lmodern}
\fi

\usepackage{charter}

% Some settings:
\raggedright
\AtBeginEnvironment{adjustwidth}{\partopsep0pt} % remove space before adjustwidth environment
\pagestyle{empty} % no header or footer
\setcounter{secnumdepth}{0} % no section numbering
\setlength{\parindent}{0pt} % no indentation
\setlength{\topskip}{0pt} % no top skip
\setlength{\columnsep}{0.15cm} % set column seperation
\pagenumbering{gobble} % no page numbering

\titleformat{\section}{\needspace{4\baselineskip}\bfseries\large}{}{0pt}{}[\vspace{1pt}\titlerule]

\titlespacing{\section}{
    % left space:
    -1pt
}{
    % top space:
    0.3 cm
}{
    % bottom space:
    0.2 cm
} % section title spacing

\renewcommand\labelitemi{$\vcenter{\hbox{\small$\bullet$}}$} % custom bullet points
\newenvironment{highlights}{
    \begin{itemize}[
        topsep=0.10 cm,
        parsep=0.10 cm,
        partopsep=0pt,
        itemsep=0pt,
        leftmargin=0 cm + 10pt
    ]
}{
    \end{itemize}
} % new environment for highlights


\newenvironment{highlightsforbulletentries}{
    \begin{itemize}[
        topsep=0.10 cm,
        parsep=0.10 cm,
        partopsep=0pt,
        itemsep=0pt,
        leftmargin=10pt
    ]
}{
    \end{itemize}
} % new environment for highlights for bullet entries

\newenvironment{onecolentry}{
    \begin{adjustwidth}{
        0 cm + 0.00001 cm
    }{
        0 cm + 0.00001 cm
    }
}{
    \end{adjustwidth}
} % new environment for one column entries

\newenvironment{twocolentry}[2][]{
    \onecolentry
    \def\secondColumn{#2}
    \setcolumnwidth{\fill, 4.5 cm}
    \begin{paracol}{2}
}{
    \switchcolumn \raggedleft \secondColumn
    \end{paracol}
    \endonecolentry
} % new environment for two column entries

\newenvironment{threecolentry}[3][]{
    \onecolentry
    \def\thirdColumn{#3}
    \setcolumnwidth{, \fill, 4.5 cm}
    \begin{paracol}{3}
    {\raggedright #2} \switchcolumn
}{
    \switchcolumn \raggedleft \thirdColumn
    \end{paracol}
    \endonecolentry
} % new environment for three column entries

\newenvironment{header}{
    \setlength{\topsep}{0pt}\par\kern\topsep\centering\linespread{1.5}
}{
    \par\kern\topsep
} % new environment for the header

\newcommand{\placelastupdatedtext}{% \placetextbox{<horizontal pos>}{<vertical pos>}{<stuff>}
  \AddToShipoutPictureFG*{% Add <stuff> to current page foreground
    \put(
        \LenToUnit{\paperwidth-2 cm-0 cm+0.05cm},
        \LenToUnit{\paperheight-1.0 cm}
    ){\vtop{{\null}\makebox[0pt][c]{
        \small\color{gray}\textit{Last updated in September 2024}\hspace{\widthof{Last updated in September 2024}}
    }}}%
  }%
}%

% save the original href command in a new command:
\let\hrefWithoutArrow\href

% new command for external links:


\begin{document}
    \newcommand{\AND}{\unskip
        \cleaders\copy\ANDbox\hskip\wd\ANDbox
        \ignorespaces
    }
    \newsavebox\ANDbox
    \sbox\ANDbox{$|$}

    \begin{header}
        \fontsize{25 pt}{25 pt}\selectfont Arthur Correia

        \vspace{5 pt}

        \normalsize
        \mbox{Bairro Morro do Espelho, São Leopoldo - RS}%
        \kern 5.0 pt%
        \AND%
        \kern 5.0 pt%
        \mbox{\hrefWithoutArrow{mailto:arthurcorreia78@gmail.com}{arthurcorreia78@gmail.com}}%
        \kern 5.0 pt%
        \AND%
        \kern 5.0 pt%
        \mbox{\hrefWithoutArrow{tel:+55 (51)9 8314-5323}{+55 (51) 9 8314-5323}}%
        \kern 5.0 pt%
        \AND%
        \kern 5.0 pt%

        \normalsize
        \mbox{Data de Nascimento: 23/08/2003}%
        \kern 5.0 pt%
        \AND%
        \kern 5.0 pt%
        
    
        
    \end{header}

    \vspace{5 pt - 0.3 cm}


    \section{Objetivo Profissional}



        
        \begin{onecolentry}
            \href{https://rendercv.com}{Busco uma oportunidade no setor de auxiliar de produção, onde possa aplicar minhas experiencias na área de produção e contribuir para o crescimento da empresa.}
        \end{onecolentry}

        \vspace{0.2 cm}




    
    \section{Qualificações}

    \begin{onecolentry}
        \begin{highlightsforbulletentries}


        \item Experiencia em auxiliar de produção na area industrial


        \end{highlightsforbulletentries}
    \end{onecolentry}

    
    \section{Experiencia}



        
        \begin{twocolentry}

        
            \textbf{Auxiliar de Produção}, IBM Militar -- São Leopoldo, RS\end{twocolentry}

        \vspace{0.10 cm}
        \begin{onecolentry}
            \begin{highlights}
                \item Identificação e separação de peças com possíveis defeitos, contribuindo para a manutenção do controle de qualidade. 
                \item Ajudante na organização dos materiais e equipamentos, garantindo a disponibilidade e eficiência na linha de produção. 
                \item Foi realizado auxilio de medidas e padrões de corte, seguindo fichas técnicas e modelos, para atender aos requisitos de produção. 
            \end{highlights}
        \end{onecolentry}


        \vspace{0.2 cm}

        \begin{twocolentry}{
            June 2003 – Aug 2003
        }


        \vspace{0.10 cm}



        
        \begin{samepage}
            \begin{twocolentry}{
                Jan 2004
            }
               

            \vspace{0.10 cm}
            
            \begin{onecolentry}

                \vspace{0.10 cm}
                
 \section{Formação Acadêmica}

\begin{onecolentry}
    \begin{itemize}
        \item \textbf{Ensino Médio Completo} - Escola Estadual de Educação Professor Pedro Schneider  
        \textit{Período: Janeiro de 2019 - Dezembro de 2022}
        
        \item \textbf{Cursando Curso Técnico em Desenvolvimento de Sistemas} - Serviço Nacional de Aprendizagem Comercial (SENAC)  
        \textit{Período: Setembro de 2024 - Dezembro de 2026}
    \end{itemize}
\end{onecolentry}


            

\section{Cursos}

\begin{onecolentry}
    \begin{itemize}
        \item \textbf{Informática Básica} - Senac São Leopoldo  
        \textit{Período: Março de 2018 – Setembro de 2018}

        \item \textbf{Análise de Sistemas (Cursando)} - Senac São Leopoldo  
        \textit{Início: Atualmente}
    \end{itemize}
\end{onecolentry}

            


    
    

    
    \section{Habilidades}



        
        \begin{twocolentry}{
            \href{https://github.com/sinaatalay/rendercv}{github.com/name/repo}
        }

        \vspace{0.10 cm}
        \begin{onecolentry}
            \begin{highlights}
                \item Conhecimento Avançado em Pacote Office: World, Office e Power Point
                \item Conhecimento Avnaçado no Pacote Adobbe: Photoshop e Premier
            \end{highlights}
        \end{onecolentry}


        \vspace{0.2 cm}

        \begin{twocolentry}{
            \href{https://github.com/sinaatalay/rendercv}{github.com/name/repo}
        }
        

        \vspace{0.2 cm}

        \begin{twocolentry}{
            2002
        }



    
    \section{Idiomas}

\begin{onecolentry}
    \begin{itemize}
        \item \textbf{Português} - Nativo
        \item \textbf{Espanhol} - Básico
        \item \textbf{Inglês} - Médio
    \end{itemize}
\end{onecolentry}



    

\end{document}